\documentclass[a4paper,12pt]{article}
\usepackage[utf8x]{inputenc}
\usepackage{ucs}
\usepackage[english]{babel}
\usepackage{amsfonts}
\usepackage{amssymb}
\usepackage{amsmath}
\usepackage{hyperref}

\title{Solving CAPTCHAs}
\author{Mihai Maruseac, Lucian Mogoșanu, Sofia Neață, Adrian Șendroiu}
\date{March 2012}

% TODO: use a different template/layout for paper/journal
% TODO: restructure tex if document has the change of getting big
\begin{document}

\maketitle

The repository for this project can be found on GitHub
\footnote{\url{https://github.com/mihaimaruseac/ssl}}.

\section*{Project motivation}

The CAPTCHA (Completely Automated Public Turing test to tell
Computers and Humans Apart) mechanism is a measure widely used on
the World Wide Web to provide tests solvable only by human users.
It is commonly used as a means to prevent problems such as automated
spam, website registration, collection of e-mail addresses or abuse
of services such as online voting.

Usually CAPTCHAs consist of a challenge-response test in which the
user is provided with an image containing a small number of words
and is asked to type them back. This kind of test is in theory
easily solvable by a human agent, while the problem of recognizing the
text in an image is not trivial from the point of view of algorithms.
Moreover, the CAPTCHA usually contains noise and/or distorted text,
making the use techniques such as Optical Character Recognition
difficult. This is similar to the manner in which one-way functions
are used in cryptographic algorithms.

Our project aims to use Machine Learning for the purpose of solving 
CAPTCHAs. There are several approaches to this. One approach would
involve using a free/open source CAPTCHA generator to generate
a set of training and/or test examples and try to improve on
various algorithms known to work well on this problem. Testing
against well-known CAPTCHA systems such as Google's reCAPTCHA
\footnote{\url{http://www.google.com/recaptcha}} could also give
us a good idea about the performance of our algorithm(s), as
well as the effectiveness of CAPTCHA as a security mechanism.

\section*{State of The Art}

CAPTCHA tests can be found in various forms on the Internet, ranging
from short audio streams to images. Thus previous research on the
matter has been focused more or less on all these aspects. For
example Tam et al. \cite{VonAhn_Tam_Hyde_Simsa_2009} use a technique
similar to a Fast Fourier Transform (FFT), along with other methods,
to break audio CAPTCHAs. Merler and Jacob \cite{Merler:2009} attempt 
to solve CAPTCHAs that consist of a combination of letters and
images from given categories, using VidoopCAPTCHA
\footnote{\url{http://vidoop.com/captcha/}} to generate data and
image classifiers based on Support Vector Machines (SVM) to obtain
a model of the CAPTCHA.

However the most common type of CAPTCHA are those that are based
on text. The text is usually presented in a distorted form
and accompanied by noise so that it can't be solved by using OCR
techniques. Google's reCAPTCHA uses audio as well as text as
a challenge for the user, at the same time using the user's
responses to provide a basis for digitization of books, newspapers
and old radio shows. von Ahn et al.'s paper \cite{vonAhn12092008}
gives a short presentation of how this is accomplished.

A large number of researchers have worked on breaking the EZ-Gimpy
and Gimpy \footnote{\url{http://www.captcha.net/captchas/gimpy/}}
CAPTCHA generators used by Yahoo. From these, an interesting
approach is that of Mori and Malik
\cite{Mori:2003:ROA:1965841.1965858}, trying to recognize letters
by separating them from the CAPTCHA clutter using Shape Contexts
 - this is basically done by trying to match candidate shapes
 against a predefined database of objects.


\vskip 0.2in
\bibliographystyle{plain}
\bibliography{bibliography}

\end{document}
